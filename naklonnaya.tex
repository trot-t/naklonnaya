\documentclass{article}

\usepackage[margin=20mm]{geometry}
\geometry{
        a4paper,
        total={165mm,247mm},
        top=30mm,
        bottom=20mm
%       evenmargin=25mm,
%       oddsidemargine=20mm
%       left=20mm
}


\usepackage[T2A]{fontenc}
\usepackage[utf8]{inputenc}
\usepackage[english,russian]{babel}
\usepackage{tikz}
\usepackage[european,cuteinductors,smartlabels]{circuitikz}
\usepackage{setspace}
\usepackage{amsmath}
\usepackage{accents}
\usepackage{amssymb,graphicx}

\makeatletter
\renewcommand*\dot[1]{%
%	\accentset{\cc@dot}{#1}%	
	\accentset{{\cc@style{\cdot}}}{#1}
}


\title{Уравнения Даламбера для тела на наклонной плоскости}
\author{Прокшин А.Н. }
% Конец преамбулы
%https://tex.stackexchange.com/questions/268612/how-to-write-a-text-along-a-circle
%https://tex.stackexchange.com/questions/306457/how-to-bend-a-mathematical-formula-around-an-arc-or-circle
\usetikzlibrary{decorations.text,fit,chains,calc,shapes.geometric,intersections}
\begin{document}
\pagenumbering{gobble}
%\maketitle
\begin{tikzpicture}[function 4/.style={
   draw opacity=0.1, 
   rotate=0,
   postaction={decorate,
               decoration={
                         raise=-1.5ex,
                         text along path,
                         %reverse path,
                         text align={fit to path stretching spaces},
                         text={%
				 |\color{red}| ${x_2} = {\tg\alpha} {x_1}$ } 
                              }
                        }
                }
    ]
\newcommand{\alfa}{35}
\newcommand{\D}{6}
\newcommand{\F}{3}
\newcommand{\Q}{2}
\newcommand{\Qq}{\Q/sin(\alfa)}
\newcommand{\QQ}{\Q/sin(\alfa)}
\draw[thin, ->] (-5.5,0) -- (6,0) node[right] {$x_1$};
	\draw[thin, ->] (0,-3.8) -- (0,6) node[left] {$x_2$} ;
	\draw ({-\D*cos(\alfa)},{-\D*sin(\alfa)}) --  (0,0) ({\QQ*cos(\alfa)},{\QQ*sin(\alfa)}) -- ({\D*cos(\alfa)},{\D*sin(\alfa)}); 
	\draw[ultra thick,red,->,>=latex] (0,0) -- (0,{-\F}) node[midway,right] {$\vec{F}$};
%	\draw[ postaction={decorate,decoration={text effects along path, text={  x },text align=center, text effects/.cd, text along path, every character/.style={fill=yellow, yshift=5mm}}}] 
%	({\D*cos(\alfa)/2},{\D*sin(\alfa)/2}) -- ({\D*cos(\alfa)},{\D*sin(\alfa)});
	\path[function 4] ({0.8*\D*cos(\alfa)},{0.8*\D*sin(\alfa)}) -- ({1.2*\D*cos(\alfa)},{1.2*\D*sin(\alfa)});

	\draw[blue,->,>=latex, ultra thin] (0,0) -- ({\Q*(-sin(\alfa))},{\Q*cos(\alfa)}) node[left] {$\vec{q}_1$};
	\draw[blue,->, ultra thin] (0,0) -- (0,{\Q}) node[right] {$\vec{q}_2$};
	\draw[magenta,->, ultra thin] (0,0) -- ({-\Qq},0) node[above] {$\vec{q}^1$};
	\draw[magenta,->] (0,0) -- ({\QQ*cos(\alfa)},{\QQ*sin(\alfa)}) node[left] {$\vec{q}^2$};
%%%%%%%
	\newcommand{\zx}{2}
	\newcommand{\zy}{6}
	\draw[->,>=latex'] (0,0) -- ({\zx},{\zy}) node[right] {$F^i,q^i(x_1,x_2)$}; % z= zx + zy
	\draw[help lines,densely dotted] ({\Q*(-sin(\alfa))},{\Q*cos(\alfa)})  -- ({\Q*(-sin(\alfa)) + \zx/6},{\Q*cos(\alfa) + \zy/6});
	\draw[help lines, blue, thin, dashed,->,>=latex'] (0,0) -- ({\Q*(-sin(\alfa)) + \zx/6},{\Q*cos(\alfa) + \zy/6}) ;
	\draw[help lines,thin]  ({\Q*(-sin(\alfa)) + \zx/6},{\Q*cos(\alfa) + \zy/6}) --  ({2*(\Q*(-sin(\alfa)) + \zx/6)},{2*(\Q*cos(\alfa) + \zy/6)}) ;
	\draw[help lines, loosely dashdotted] ({\Q*(-sin(\alfa))},{\Q*cos(\alfa)}) -- ({2*\Q*(-sin(\alfa))},{2*\Q*cos(\alfa)});
	\draw[help lines,densely dotted]  ({2*\Q*(-sin(\alfa))},{2*\Q*cos(\alfa)}) -- ({2*(\Q*(-sin(\alfa)) + \zx/6)},{2*(\Q*cos(\alfa) + \zy/6)}) ;
% нижний
	\newcommand{\lowx}{(-4*cos(\alfa)}
	\newcommand{\lowy}{(-4*sin(\alfa)}
	\draw[help lines, ultra thin] ({\lowx},{\lowy}) -- ({\lowx + 2*\Q*(-sin(\alfa))},{\lowy + 2*\Q*cos(\alfa)}) node (Alow) {};
	\draw[help lines, ultra thin, name path=line 1] ({\lowx},{\lowy}) -- ({\lowx + 2*(\Q*(-sin(\alfa)) + \zx/6)},{\lowy + 2*(\Q*cos(\alfa) + \zy/6)}) node (A) {};
        \renewcommand{\lowx}{(2.5*cos(\alfa)}
        \renewcommand{\lowy}{(2.5*sin(\alfa)}
	\draw[help lines, ultra thin, loosely dashdotted] ({\lowx},{\lowy}) -- ({\lowx + 2*\Q*(-sin(\alfa))},{\lowy + 2*\Q*cos(\alfa)}) node (Blow) {};
	\draw[help lines, ultra thin] ({\lowx},{\lowy}) -- ({\lowx + 2*(\Q*(-sin(\alfa)) + \zx/6)},{\lowy + 2*(\Q*cos(\alfa) + \zy/6)}) node (B) {};
	\draw[help lines] (A.center) -- (B.center);
	\path[name path= line 2] (Alow.center) -- (Blow.center); 
	\path [name intersections={of=line 1 and line 2, by=E}];
	\draw[help lines] (Alow.center) -- (E);
        \draw[help lines, loosely dashdotted] (E) -- (Blow.center);	
\end{tikzpicture}

Наклонная плоскость описывается уравнением (в декартовой системе ковариантные и контравариантные координаты вектора совпадают)
\begin{equation}
x_1\cdot\tg\alpha = x_2 + const
\label{nakl}
\end{equation}


\begin{equation}
\left\{
%\begingroup \renewcommand\baselinestretch{1.6}\selectfont	
\begin{onehalfspacing}
	\begin{array}{l}	
		F^1(x_1,x_2) = - \sin\alpha\cdot x_1 + \cos\alpha\cdot x_2 + q^1_0 = 0 = q^1(x_1,x_2) - \textcyrillic{уравнение связи,} \\ 
%\\
		F^2(x_1,x_2) = x_2 + q^2_0 = q^2(x_1,x_2) - \textcyrillic{свободная переменная}
	\end{array}
\end{onehalfspacing}	
%\endgroup
\right.
\label{initial}
\end{equation}

%$$
%\dddot{x}
%$$
%\makeatletter
%\renewcommand*\dddot[1]{
%\placeaccent{\acc@dot}{#1}
%}

Импульсы в системе $x_1,x_2$ обозначим через $\xi_i = m \dot{x}_i$, силы в системе обозначим через $X_i$

Уравнения движения запишутся:
\begin{equation} 
\left\{%\begin{doublespacing}
\begingroup \renewcommand\baselinestretch{2.2}\selectfont	
\begin{array}{l}
	{\displaystyle	\frac{d\xi_1}{dt} = X_1 + \lambda_1 \frac{\partial F^1}{\partial x_1} } - \textcyrillic{частные производные только по связи } F^1\\ 
	{\displaystyle \frac{d\xi_2}{dt} = X_2 + \lambda_1 \frac{\partial F^1}{\partial x_2} }
\end{array}%\end{doublespacing}
\endgroup
\right.
\label{motion_equation}
\end{equation}

%Частные производные только по связи.

Производные по времени:
$$
\sum\limits_{i=1}^2 \frac{\partial F^k}{\partial x_i}\dot{x}_i = 
\left\{\begin{onehalfspacing}
%	\begin{spacing}{1.0}
	\begin{array}{l}
		\dot{q}^k, - \textcyrillic{свободная переменная} \\
		0 - \textcyrillic{связь}\end{array}
%	\end{spacing}
	\end{onehalfspacing}
		\right.
$$
В развернутом виде для выбранной системы координат:
\begin{equation}
\left\{\begin{doublespacing}
%       \begin{spacing}{1.0}
	\begin{array}{ccc}
		\frac{\partial q^1}{\partial x_1} \dot{x_1} +  \frac{\partial q^1}{\partial x_2} \dot{x_2} &=& 0 \\
		\frac{\partial q^1}{\partial x_1} \dot{x_1} +  \frac{\partial q^1}{\partial x_2} \dot{x_2} &=& \dot{q}^2
	\end{array}
        \end{doublespacing}
\right.		
\end{equation}
	И если уравнения для координат (\ref{initial}) является <<точечным>> преобразованием, то уравнение для скоростей является линейным.

в наших обозначениях 

$$
	\frac{\partial F^1}{\partial x_1}\dot{x}_1  + \frac{\partial F^1}{\partial x_2} \dot{x}_2 = - \sin\alpha \:\dot{x}_1 + \cos\alpha \:\dot{x}_2 = \dot{q}^1 = 0
-  \textcyrillic{из уравнения связи}
$$
$$
	\dot{x}_2 = \dot{q}^2 - \textcyrillic{из уравнения для свободной переменной}
$$

Из этих уравнений определяем $\dot{x}_1$ и $\dot{x}_2$

$$
\left\{\begin{onehalfspacing}
\begin{array}{l}
	\dot{x}_1 = \ctg\alpha\:\dot{q}^2 \\
	\dot{x}_2 = \dot{q}^2
\end{array}
\end{onehalfspacing}
\right.
$$

$$
\left\{\begin{doublespacing}
\begin{array}{lll}
	\frac{\partial x_1}{\partial q^1} \dot{q}^1 + \frac{\partial x_1}{\partial q^2}  \dot{q}^2 &=& \dot{x}_1 \\
	\frac{\partial x_2}{\partial q^1} \dot{q}^1 + \frac{\partial x_2}{\partial q^2}  \dot{q}^2 & = & \dot{x}_2
\end{array}
\end{doublespacing}
\right.	
$$

Из независимости работы от системы координат
$$
\delta A = \sum\limits_{i=1}^2 X_i \delta x_i = \sum\limits_\textcyrillic{
по свободным
		} Q_i \delta q^i 
$$
и затем перейдя к скоростям 

$$ 
	\sum\limits_\textcyrillic{ по свободным } Q_i \delta \dot{q}^i = \sum\limits_{i=1}^2 X_i \delta \dot{x}_i
$$

$$
	Q_2 \delta \dot{q}^2 = X_1 \ctg\alpha\:\delta\dot{q}^2 + X_2 \delta\dot{q}^2 
$$

подставляя $X_1 = 0$ и $X_2 = -F$

$$
Q_2 = -F = Q^2
$$
Заметим, что вектор, лежащий на оси, имеет совпадающие ковариантные и контравариантные координаты.

Найдем $\lambda_1$

$$
\left\{\begin{onehalfspacing}
\begin{array}{l}
\dot{\xi}_1 = X_1 + \lambda_1 \sin\alpha \\
\dot{\xi}_2 = X_2 - \lambda_1 \cos\alpha
\end{array} 	
\end{onehalfspacing}\right. =
\left\{\begin{onehalfspacing}
\begin{array}{l}
\dot{\xi}_1 = 0 + \lambda_1 \sin\alpha \\
\dot{\xi}_2 = -F - \lambda_1 \cos\alpha
\end{array}
\end{onehalfspacing}\right.
$$

Из уравнения связи 
$$
{\displaystyle \dot{\xi}_2/\tg\alpha = \lambda_1\sin\alpha \Rightarrow \dot{\xi}_2 = \lambda_1 \frac{\sin^2\alpha}{\cos\alpha}}
$$

$$
\dot{\xi}_2 = X_2 - \lambda_1 \cos\alpha 
$$

$$
\lambda_1\left(\frac{\sin^2\alpha}{\cos\alpha} + \cos\alpha\right) = X_2
$$

$$
\lambda_1 \frac{1}{\cos\alpha} = X_2
$$

\begin{equation}
	\lambda_1 =\underbrace{X_2\cos\alpha}_\textcyrillic{это сила реакции опоры}	
\end{equation}	

Подставляя найденную $\lambda_1$ в уравнения движения (\ref{motion_equation}) получаем:

\begin{equation}
\left\{%\begin{doublespacing}
\begingroup \renewcommand\baselinestretch{2.2}\selectfont
\begin{array}{l}
	{\displaystyle  \frac{d\xi_1}{dt} = 0 + X_2\cos\alpha \cdot \sin\alpha  = -F \cos\alpha \cdot \sin\alpha }\\
        {\displaystyle \frac{d\xi_2}{dt} = X_2 - X_2\cos\alpha \cdot \cos\alpha  = X_2 \sin^2\alpha = F\sin^2\alpha}
\end{array}%\end{doublespacing}
\endgroup
\right.  
\end{equation}

Импульсы преобразуются по формуле аналогичной силе $Q$
$$
p_2 = \xi_1 \ctg\alpha + \xi_2
$$

	Уравнения движения (\ref{motion_equation}) преобразуются к виду

$$
	\frac{\partial p_2}{\partial t} = Q_2 + \lambda_1  \frac{\partial F^1}{\partial x_1} \cdot \ctg\alpha  + \lambda_1\frac{\partial F^1}{\partial x_2} = Q_2 = -F
$$
все члены с $\lambda_1$ обнуляются.
\end{document}
